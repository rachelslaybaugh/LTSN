\documentclass[preprint,12pt,authoryear]{elsarticle}

\usepackage{amsthm} % theorems AMS style
\usepackage{amssymb} % math symbols
\usepackage{amstext} % typeset text in math environments
\usepackage{amsmath} % AMS math­e­mat­i­cal fa­cil­i­ties
\usepackage{bm} % bold symbols in math
\usepackage{graphicx} % enhanced support for graphics
\usepackage{epstopdf} % con­vert EPS to 'en­cap­su­lated' PDF us­ing ghostscript
\usepackage{subcaption} % support for subcaptions
\usepackage{float}
\usepackage{tikz}
\usepackage{lineno,hyperref}
\usepackage[capitalize]{cleveref} % in­tel­li­gent cross-ref­er­enc­ing
\usepackage[makeroom]{cancel} %cancel terms
\usepackage{multirow}


\modulolinenumbers[5]

% Adjusting page layouts and margins
\pagestyle{plain}
\renewcommand{\baselinestretch}{1.0} 
\setlength{\topmargin}{-0.5in} 
\setlength{\oddsidemargin}{0.in} 
\setlength{\evensidemargin}{0.in} 
\setlength{\textwidth}{6.5in} 
\setlength{\textheight}{9.0in} 
\setlength{\parskip}{0pt}    

%\renewcommand{\theequation}{\arabic{section}.\arabic{equation}}

% Adjusting cleveref package
\crefalias{subequation}{equation} % subequations counter = equations counter
\crefalias{eqnarray}{equation} % subequations counter = equations counter
\crefformat{pluraleq}{Eqs.~(#2#1#3)} % defining "plural" equations  
\Crefformat{pluraleq}{Equations~(#2#1#3)} % defining "plural" equations  

\newcommand{\SN}{S$_N$}
\renewcommand{\vec}[1]{\bm{#1}} %vector is bold italic
\newcommand{\vd}{\bm{\cdot}} % slightly bold vector dot
\newcommand{\grad}{\vec{\nabla}} % gradient
\newcommand{\ud}{\mathop{}\!\mathrm{d}} % upright derivative symbol
\def\bal#1\nal{\begin{align}#1\end{align}}
\def\bala#1\nala{\begin{align*}#1\end{align*}}
\def\bsub#1\nsub{\begin{subequations}#1\end{subequations}}
\newcommand{\f}{\frac}
\newcommand{\ux}{{\bm x}}
\newcommand{\un}{{\bm n}}
\newcommand{\unab}{{\bf \nabla}}
\newcommand{\bg}{\big>}
\newcommand{\bl}{\big<}
\newcommand{\su}{\big< s\big>}
\newcommand{\sd}{\big< s^2\big>}
\newcommand{\st}{\big< s^3\big>}
\newcommand{\sq}{\big< s^4\big>}
\renewcommand{\sc}{\big< s^5\big>}
\renewcommand{\ss}{\big< s^6\big>}
\newcommand{\sue}{\big< s\big>_\epsilon}
\newcommand{\sde}{\big< s^2\big>_\epsilon}
\newcommand{\ste}{\big< s^3\big>_\epsilon}
\newcommand{\sqe}{\big< s^4\big>_\epsilon}
\newcommand{\sce}{\big< s^5\big>_\epsilon}
\newcommand{\sse}{\big< s^6\big>_\epsilon}
\newcommand{\wsa}{\widehat\Sigma_a}
\newcommand{\wst}{\widehat\Sigma_t}
\newcommand{\wq}{\widehat Q}
\newcommand{\ep}{\varepsilon}
\newcommand{\vi}{{\varphi}}
\newcommand{\uom}{{\bf \Omega}}
\newcommand{\setb}{\mathcal{B}}

\journal{TBD - Annals? JQSRT? JCP? }

\bibliographystyle{elsarticle-harv.bst}\biboptions{authoryear}
%%%%%%%%%%%%%%%%%%%%%%%

\begin{document}

\begin{frontmatter}

\title{A Spectral Approach to the Nonclassical Transport Equation\\
(tentative title)}

%% Group authors per affiliation:
\author[ucb]{R. Vasques\corref{cor1}}
\author[ufrgs]{C.F. Segatto \fnref{segatto}}
\author[ucb]{R.N. Slaybaugh\fnref{slaybaugh}}

\address[ucb]{University of California, Berkeley, Department of Nuclear Engineering, 4155 Etcheverry Hall \\ Berkeley, CA 94720-1730}
\address[ufrgs]{UFRGS - Federal University of Rio Grande do Sul, Av. Osvaldo Aranha 99, 90046-900\\ Porto Alegre, RS, Brazil}

\cortext[cor1]{Corresponding author: richard.vasques@fulbrightmail.org; Tel: (510) 340 0930\\
Postal address: University of California, Berkeley, Department of Nuclear Engineering, 4103 Etcheverry Hall, Berkeley, CA 94720-1730}
\fntext[cegatto]{cynthia.segatto@ufrgs.br}
\fntext[slaybaugh]{slaybaugh@berkeley.edu}

\begin{abstract}

These notes describe an approach to manipulate the nonclassical transport equation into a classical form that can be numerically solved through traditional approaches. 
The approach uses a combination of the spectral method and source iteration to eliminate the $s$-dependence.
We use the LTS$_N$ method to solve the resulting equation in a 1-D system. 
 
\end{abstract}

\begin{keyword}
tbd \sep tbd 
\end{keyword}

\end{frontmatter}

\section{Introduction}\label{sec1}
\setcounter{section}{1}
\setcounter{equation}{0} 

\section{Nonclassical Transport Equation}\label{sec2}
\setcounter{section}{2}
%\setcounter{equation}{0} 

Consider the one-speed nonclassical transport equation with isotropic scattering given by
\bsub\label[pluraleq]{1}
\bal
\f{\partial }{\partial s}\Psi(\ux,\uom,s) + \uom\cdot\unab&\Psi(\ux,\uom,s) + \Sigma_t(\uom,s)\Psi (\ux,\uom,s) = \label{1a}\\
& \f{\delta(s)}{4\pi}\left[\int_{4\pi}\int_0^{\infty}c\Sigma_t(\uom',s')\Psi(\ux,\uom',s')d\Omega' ds' + Q(\ux)\right],\quad \ux\in V,\nonumber
\nal
subject to the vacuum boundary condition
\bal
%\Psi(0,\mu,s) &= \delta(s)f(\mu), \quad \mu > 0,\\
%\Psi(X,\mu,s) &= \delta(s)g(\mu), \quad \mu < 0.
\Psi(\ux_b,\uom,s) &= 0, \quad \un\cdot\uom< 0, \quad \ux_b\in\partial V.
%\Psi(X,\mu,s) &= 0, \quad \mu < 0.
\nal
\nsub
Here, $\ux=(x,y,z)$, $\uom = (\Omega_x,\Omega_y,\Omega_z)$, $s$ describes the free-path of a particle, $\Psi$ is the nonclassical angular flux, $c$ is the scattering ratio, and $Q$ is an isotropic source. 
The angular-dependent nonclassical total cross section $\Sigma_t(\uom,s)$ satisfies
\bal
p(\uom,s) = \Sigma_t(\uom,s)e^{-\int_0^s \Sigma_t(\uom,s')ds'},
\nal
where $p(\uom,s)$ is the free-path distribution function in the direction $\uom$.

It is useful to work with \cref{1a} in an equivalent ``initial value'' form:
\bsub\label[pluraleq]{3}
\bal
&\f{\partial }{\partial s}\Psi(\ux,\uom,s) + \uom\cdot\unab\Psi(\ux,\uom,s) + \Sigma_t(\uom,s)\Psi (\ux,\uom,s) = 0,\\
& \Psi(\ux,\uom,0)=\f{1}{4\pi}\left[\int_{4\pi}\int_0^{\infty}c\Sigma_t(\uom',s')\Psi(\ux,\uom',s')d\Omega' ds' + Q(\ux)\right].
\nal
\nsub
Let us define $\psi$ such that
\bal\label{4}
\Psi(\ux,\uom,s)\equiv \psi(\ux,\uom,s)e^{-\int_0^s \Sigma_t(\uom,s')ds'}.
\nal
We can now rewrite the nonclassical problem as
\bsub\label[pluraleq]{5}
\bal
&\f{\partial }{\partial s}\psi(\ux,\uom,s) + \uom\cdot\unab\psi(\ux,\uom,s) = 0,\label{5a}\\
& \psi(\ux,\uom,0)=\f{1}{4\pi}\left[\underbrace{\int_{4\pi}\int_0^{\infty}c p(\uom',s')\psi(\ux,\uom',s')d\Omega' ds'}_{S(\ux)} + Q(\ux)\right] = \f{S(\ux)}{4\pi}+\f{Q(\ux)}{4\pi}, \label{5b}\\
%&\psi(0,\mu,s) = \delta(s)f(\mu), \quad \mu > 0,\\
%&\psi(X,\mu,s) = \delta(s)g(\mu), \quad \mu < 0.
&\psi(\ux_b,\uom,s) = \quad \un\cdot\uom< 0, \quad \ux_b\in\partial V.
%&\psi(X,\mu,s) = 0, \quad \mu < 0.
\nal
\nsub

We will take advantage of the initial value form of \cref{5} by using source iteration. We define
\bal\label{6}
\psi(\ux,\uom,s) = \sum_{k=0}^K \psi^{(k)}(\ux,\uom,s),
\nal
where $\psi^{(k)}$ represents particles in the angular flux that have undergone \textit{exactly} $k$ collisions.
It is easy to see that $\psi^{(k)}$ satisfies
\bsub\label[pluraleq]{7}
\bal
&\f{\partial }{\partial s}\psi^{(k)}(\ux,\uom,s) + \uom\cdot\unab\psi^{(k)}(\ux,\uom,s) = 0,\label{7a}\\
& \psi^{(k)}(\ux,\uom,0)= F^{(k)}(\ux) = \left\{\begin{array}{cc}
\f{Q(\ux)}{4\pi}, & k=0,\\
\f{S^{(k-1)}(\ux)}{4\pi}, & k > 0,
\end{array}
\right., \label{7b}\\
%&\psi^{(k)}(0,\mu,s) = \left\{\begin{array}{ccc}
%\delta(s)f(\mu), & \mu > 0, & k=0,\\
%0, & \mu > 0, & k > 0,
%\end{array}
%\right.\\
&\psi^{(k)}(\ux_b,\uom,s) = 0,\quad \un\cdot\uom< 0, \quad \ux_b\in\partial V.
%&\psi^{(k)}(X,\mu,s) = \left\{\begin{array}{ccc}
%\delta(s)g(\mu), & \mu < 0, & k=0,\\
%0, & \mu < 0, & k > 0,
%\end{array}
%\right.
%&\psi^{(k)}(X,\mu,s) = 0, \quad \mu <0,
\nal
\nsub
where $S^{(k-1)}(\ux) = \int_{4\pi}\int_0^{\infty}c p(\uom',s')\psi^{(k-1)}(\ux,\uom',s')d\Omega' ds'$.

The idea in these notes is to use the spectral method to eliminate the $s$-dependence in \cref{7}.

\section{The Spectral Method}\label{sec3}
\setcounter{section}{3}
%\setcounter{equation}{0} 

To eliminate the dependence on $s$, we approximate $\psi^{(k)}$ by a truncated series of Laguerre polynomials $\{ L_0(s), L_1(s), ..., L_M(s)\}$ such that
\bal\label{8}
\psi^{(k)}(\ux,\uom,s) = \sum_{m=0}^{M} \psi^{(k)}_m(\ux,\uom)L_m(s).
\nal  
The Laguerre polynomials are orthogonal with respect to the weight function $e^{-s}$; that is,
\bal
\int_0^{\infty} e^{-s}L_j(s)L_m(s)ds =
\left\{\begin{array}{cc}
0 & j\neq m,\\
1 & j = m.
\end{array}
\right.
\nal
We multiply \cref{7a} by $e^{-s}L_m(s)$ and operate on it by $\int_0^\infty (\cdot)ds$ (truncating in $M$) to obtain
\bal
\uom\cdot\unab\psi^{(k)}_m(\ux,\uom) = \sum_{j=m+1}^M\psi^{(k)}_j(\ux,\uom).
\nal
\Cref{7b,8} give us
\bal
\sum_{j=m+1}^M\psi^{(k)}_j(\ux,\uom) = F^{(k)}(\ux)  - \sum_{j=0}^m\psi_j^{(k)}(\ux,\uom).
\nal
We now have
\bsub\label[pluraleq]{12}
\bal
&\uom\cdot\unab\psi^{(k)}_m(\ux,\uom) + \psi^{(k)}_m(\ux,\uom) = U^{(k)}_m(\ux,\uom),\\
%&\psi^{(k)}_m(0,\mu) = \left\{\begin{array}{ccc}
%f(\mu)=e^{-s/2}\sum_{j=0}^ML_j(s), & \mu > 0, & m=0,\\
%0, & \mu > 0, & m\neq 0,
%\end{array}
%\right.\\
%&\psi_m(X,\mu) = \left\{\begin{array}{ccc}
%g(\mu), & \mu < 0, & m=0,\\
%0, & \mu < 0, & m\neq 0.
%\end{array}
%\right.\\
&\psi^{(k)}_m(\ux_b,\uom) = 0, \quad \un\cdot\uom< 0, \quad \ux_b\in\partial V,
%&\psi^{(k)}_m(X,\mu) = 0, \quad \mu<0.
\nal
\nsub
where
\bsub
\bal
U^{(k)}_0(\ux,\uom) &= F^{(k)}(\ux),\\
U^{(k)}_m(\ux,\uom) &= U^{(k)}_{m-1}(\ux,\uom) - \psi^{(k)}_{m-1}(\ux,\uom), \quad m=1,...,M.
\nal
\nsub
%Both $F^{(k)}(\ux)$ and $\psi_j^{(k)}(\ux,\uom)$ are known for $j<m$.
For each step $m$, $U_m^{(k)}$ is a known function.
\Cref{12} have the form of a \textit{classical} purely absorbing homogeneous system with a fixed source and $\Sigma_t=\Sigma_a=1$.
That is, \cref{12} can be solved using \textit{any} established homogeneous solver. 

A simple sketch of the algorithm follows:
\begin{itemize}
\item ``While'' loop: $k = 0,1,2,...$ (uses a stopping criterion)
\begin{itemize}
\item Closed loop: $m=0,1,...,M$
\begin{itemize}
\item Apply homogeneous solver to \cref{12} and obtain $\psi^{(k)}_m$
\end{itemize}
\item Check stopping criterion
\end{itemize}
\item Obtain the classical angular flux:
$$\hat\Psi(\ux,\uom) = \int_0^\infty \Psi(\ux,\uom,s)ds =  \int_0^{\infty} \left(e^{-\int_0^s\Sigma_t(\uom,s')ds'}\sum_{k=0}^{K}\sum_{m=0}^M\psi^{(k)}_m(\ux,\uom)L_m(s)\right)ds$$
\end{itemize}

In the next section we apply this approach to a 1-D problem using the LTS$_N$ method to solve \cref{12}.

\section{The LTS$_N$ Approach}\label{sec4}
\setcounter{section}{4}
%\setcounter{equation}{0} 

Let the coefficients $\mu_n$ and $\omega_n$ represent the zeros and weights of the Gauss-Legendre quadrature.
We define
\bsub
\bal
\psi^{(k)}_{m,n}(x) &= \psi^{(k)}_{m}(x,\mu_n),\\
U^{(k)}_{0,n}(x) &= U^{(k)}_{0}(x,\mu_n) = F^{(k)}(x),\\
U^{(k)}_{m,n}(x) &= U^{(k)}_{m-1,n}(x) - \psi_{m-1,n}^{(k)}(x) = U^{(k)}_{m-1}(x,\mu_n) - \psi_{m-1}^{(k)}(x,\mu_n),
\nal
where
\bal
F^{(0)}(\ux) &= \f{Q(\ux)}{2}, \\
F^{(k)}(\ux) &= \f{c}{2}\int_0^\infty \sum_{n=1}^N \left(\omega_n p_n(s)\sum_{m=1}^{M} \psi^{(k-1)}_{m,n}L_m(s)\right)ds, \quad k =1,2,...,\label{14e}
\nal
and
\bal
p_n(s) = p(\mu_n,s).
\nal
\nsub
Now, we can write the corresponding 1-D S$_N$ problem to \cref{12} as
\bsub
\bal
&\mu_n\f{\partial }{\partial x}\psi^{(k)}_{m,n}(x) + \psi^{(k)}_{m,n}(x) = U^{(k)}_{m,n}(x),\\
&\psi^{(k)}_{m,n}(0) = \psi^{(k)}_{m,N-n+1}(0), \quad n=1,...,N/2,\\
&\psi^{(k)}_{m,n}(X) = 0, \quad n=N/2+1,...,N.
\nal
\nsub
Applying the LTS$_N$ method to this problem, we obtain an analytical formulation for $\psi_{m,n}^{(k)}$:
\bal\label{16}
\psi_{m,n}^{(k)}(x) = & \sum_{i=1}^{N/2}\Bigg[\alpha(n,i)\left(e^{\gamma(i)(x-H)}+e^{\gamma(i+N/2)x}\right)\Gamma_{m,i}^{(k)} \,+ \\
& \quad  \sum_{j=1}^N \alpha(n,i)\left(\int_H^x e^{\gamma(i)(x-\tau)}U_{m,j}^{(k)}(\tau)d\tau + \int_0^x e^{\gamma(i+N/2)(x-\tau)}U_{m,j}^{(k)}(\tau)d\tau\right)\beta(i,j)\Bigg],\nonumber
\nal
where $\gamma(i)$ are the eigenvalues resulting from the diagonalization of the LTS$_N$ matrix, $\alpha(n,i)$ are the components of the eigenvector matrix, and $\beta(i,j)$ are the components of its inverse.
The coefficients $\Gamma_{m,i}^{(k)}$ are determined from the boundary conditions.

\section{Numerical Results}\label{sec5}
\setcounter{section}{5}
%\setcounter{equation}{0} 

\section{Conclusion}\label{sec6}
\setcounter{section}{6}
%\setcounter{equation}{0} 

\section*{Comments}
\setcounter{section}{4}
%\setcounter{equation}{0} 

\begin{itemize}
\item While we do not explicitly discretize $s$, the following takes place:
\begin{itemize}
\item $\psi(x,\mu,s)$ is approximated by a truncated series of Laguerre polynomials in $s$
\item The integral in $s$ described in \cref{14e} will probably need to be performed numerically
\item At the end of the algorithm, we need to calculate
$$\hat\Psi(\ux,\uom) = \int_0^\infty \Psi(\ux,\uom,s)ds =  \int_0^{\infty} \left(e^{-\int_0^s\Sigma_t(\uom,s')ds'}\sum_{k=0}^{K}\sum_{m=0}^M\psi^{(k)}_m(\ux,\uom)L_m(s)\right)ds,$$
which will also need to be performed numerically 
\end{itemize}
\item The convolution integrals in \cref{16} will \textit{probably} need to be solved numerically due to the recursiveness of the problem arising from the source term. \textbf{Is there a way to do that analitically?}
\item Due to the source iteration approach, convergence will be slow as problems become more diffusive.
\item It is not clear to me what will be the more time-consuming step. My guess is that the time to converge the source iteration will dominate in diffusive problems; for absorbing problems, I do not know.   
\item Nonclassical boundary conditions are tricky: incoming fluxes will have a $\delta(s)$. I need to think more about the best approach in those cases; it is possible that the best solution will be using the \textit{forward} nonclassical equation. That, however, is beyond the current scope.
\item To validate the method, we can apply the algorithm to solve a classical problem. In that case, $p(\mu,s) = \Sigma_te^{-\Sigma_t s}$. This will also allow us to see how efficiently the solver works.
\item After it is validated, we can apply the algorithm to the random periodic case we have been working on; after that, we can go for general stochastic mixtures.
\end{itemize}

\section*{References}

\bibliography{ANE-16}

\end{document}


\pagebreak

\noindent
{\bf FIGURE CAPTIONS}

\begin{figure}[H]
 \centering
% \includegraphics[scale=1]{fig1.eps} 
\caption{\footnotesize{DDM approach to solve the deterministic problem}}\label{fig1}
\end{figure}

\begin{figure}[H]
\centering
% \includegraphics[scale=1]{fig2.eps} 
\caption{\footnotesize{Double-DDM approach to solve the stochastic problem}}\label{fig2}
\end{figure}

\begin{figure}[H] 
 \centering
% \includegraphics[scale=1.0]{fig3.eps} 
 \caption{\footnotesize{Neutron density for a sinusoidal reactivity $\rho(t) = 0.00073\sin(t)$}}\label{fig3}
\end{figure}

\end{document}

