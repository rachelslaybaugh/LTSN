%\documentclass[preprint,12pt,authoryear]{elsarticle}
\documentclass[preprint,12pt]{elsarticle}

\usepackage{amsthm} % theorems AMS style
\usepackage{amssymb} % math symbols
\usepackage{amstext} % typeset text in math environments
\usepackage{amsmath} % AMS math­e­mat­i­cal fa­cil­i­ties
\usepackage{bm} % bold symbols in math
\usepackage{graphicx} % enhanced support for graphics
\usepackage{epstopdf} % con­vert EPS to 'en­cap­su­lated' PDF us­ing ghostscript
\usepackage{subcaption} % support for subcaptions
\usepackage{float}
\usepackage{tikz}
\usepackage{lineno,hyperref}
\usepackage[capitalize]{cleveref} % in­tel­li­gent cross-ref­er­enc­ing
\usepackage[makeroom]{cancel} %cancel terms
\usepackage{multirow} 


\modulolinenumbers[5]

% Adjusting page layouts and margins
\pagestyle{plain}
\renewcommand{\baselinestretch}{1.0} 
\setlength{\topmargin}{-0.5in} 
\setlength{\oddsidemargin}{0.in} 
\setlength{\evensidemargin}{0.in} 
\setlength{\textwidth}{6.5in} 
\setlength{\textheight}{9.0in} 
\setlength{\parskip}{0pt}    

%\renewcommand{\theequation}{\arabic{section}.\arabic{equation}}

% Adjusting cleveref package
\crefalias{subequation}{equation} % subequations counter = equations counter
\crefalias{eqnarray}{equation} % subequations counter = equations counter
\crefformat{pluraleq}{Eqs.~(#2#1#3)} % defining "plural" equations  
\Crefformat{pluraleq}{Equations~(#2#1#3)} % defining "plural" equations  

\newcommand{\SN}{S$_N$}
\renewcommand{\vec}[1]{\bm{#1}} %vector is bold italic
\newcommand{\vd}{\bm{\cdot}} % slightly bold vector dot
\newcommand{\grad}{\vec{\nabla}} % gradient
\newcommand{\ud}{\mathop{}\!\mathrm{d}} % upright derivative symbol
\def\bal#1\nal{\begin{align}#1\end{align}}
\def\bala#1\nala{\begin{align*}#1\end{align*}}
\def\bsub#1\nsub{\begin{subequations}#1\end{subequations}}
\newcommand{\f}{\frac}
\newcommand{\ux}{{\bm x}}
\newcommand{\un}{{\bm n}}
\newcommand{\unab}{{\bf \nabla}}
\newcommand{\bg}{\big>}
\newcommand{\bl}{\big<}
\newcommand{\su}{\big< s\big>}
\newcommand{\sd}{\big< s^2\big>}
\newcommand{\st}{\big< s^3\big>}
\newcommand{\sq}{\big< s^4\big>}
\renewcommand{\sc}{\big< s^5\big>}
\renewcommand{\ss}{\big< s^6\big>}
\newcommand{\sue}{\big< s\big>_\epsilon}
\newcommand{\sde}{\big< s^2\big>_\epsilon}
\newcommand{\ste}{\big< s^3\big>_\epsilon}
\newcommand{\sqe}{\big< s^4\big>_\epsilon}
\newcommand{\sce}{\big< s^5\big>_\epsilon}
\newcommand{\sse}{\big< s^6\big>_\epsilon}
\newcommand{\wsa}{\widehat\Sigma_a}
\newcommand{\wst}{\widehat\Sigma_t}
\newcommand{\wq}{\widehat Q}
\newcommand{\ep}{\varepsilon}
\newcommand{\vi}{{\varphi}}
\newcommand{\uom}{{\bf \Omega}}
\newcommand{\setb}{\mathcal{B}}

\journal{Journal of Computational Physics }

%\bibliographystyle{elsarticle-harv.bst}\biboptions{authoryear}
\bibliographystyle{elsarticle-num.bst} \biboptions{sort&compress}
%%%%%%%%%%%%%%%%%%%%%%%

\begin{document}

\begin{frontmatter}

\title{A Spectral Approach to the Nonclassical Transport Equation\\
(looking for better title)}

%% Group authors per affiliation:
\author[ucb]{R. Vasques\corref{cor1}}
\author[ufrgs]{C.F. Segatto \fnref{segatto}}
\author[ucb]{R.N. Slaybaugh\fnref{slaybaugh}}

\address[ucb]{University of California, Berkeley, Department of Nuclear Engineering, 4155 Etcheverry Hall \\ Berkeley, CA 94720-1730}
\address[ufrgs]{UFRGS - Federal University of Rio Grande do Sul, Av. Osvaldo Aranha 99, 90046-900\\ Porto Alegre, RS, Brazil}

\cortext[cor1]{Corresponding author: richard.vasques@fulbrightmail.org; Tel: (510) 340 0930\\
Postal address: University of California, Berkeley, Department of Nuclear Engineering, 4103 Etcheverry Hall, Berkeley, CA 94720-1730}
\fntext[cegatto]{cynthia.segatto@ufrgs.br}
\fntext[slaybaugh]{slaybaugh@berkeley.edu}

\begin{abstract}

These notes describe an approach to manipulate the nonclassical transport equation into a classical form that can be numerically solved through traditional approaches. 
The approach uses a combination of the spectral method and source iteration to eliminate the $s$-dependence.
We use the LTS$_N$ method to solve the resulting equation in a 1-D system. 
 
\end{abstract}

\begin{keyword}
tbd \sep tbd 
\end{keyword}

\end{frontmatter}

\section{Introduction}\label{sec1}
\setcounter{section}{1}
\setcounter{equation}{0} 

The theory of \textit{nonclassical} particle transport, which describes processes in which a particle's distance-to-collision is \textit{not} exponentially distributed, has received increased attention in the last decade.
It was originally proposed by Larsen \cite{lar_07} to describe measurements of photon path-length in the Earth's cloudy atmosphere that could not be explained by classical radiative transfer (cf. \cite{davmar_10}).
The theory has been extended over the last few years \cite{larvas_11,fragou_10,vaslar_14a,davxu_14,xudav_16} and has found applications in other areas, including neutron transport in certain types of nuclear reactors \cite{vaslar_09,vas_13,vaslar_14b}, computer graphics \cite{deo_14}, and problems involving anomalous diffusion (cf. \cite{frasun_16}).
Moreover, a similar kinetic equation has been independently derived for the periodic Lorentz gas in a series of papers by Golse (cf. \cite{gol_12}) and by Marklof and Str\" ombergsson \cite{marstr_10,marstr_11,marstr_14,marstr_15}.

The nonclassical theory requires an extended phase space that includes an extra independent variable: the free-path $s$, representing the distance traveled by a particle since its previous interaction.
The one-speed nonclassical transport equation can be written as \cite{vaslar_14a}
\bsub\label[pluraleq]{1}
\bal
\f{\partial }{\partial s}&\Psi(\ux,\uom,s) + \uom\cdot\unab\Psi(\ux,\uom,s) + \Sigma_t(\uom,s)\Psi (\ux,\uom,s) = \label{1a}\\
& \delta(s)\left[c\int_{4\pi}\int_0^{\infty}P(\uom'\cdot\uom)\Sigma_t(\uom',s')\Psi(\ux,\uom',s')d\Omega' ds' + \f{Q(\ux)}{4\pi}\right], \quad \ux \in V,\; \uom \in 4\pi, \; 0<s,\nonumber
\nal
where $\ux=(x,y,z)$, $\uom = (\Omega_x,\Omega_y,\Omega_z)$, $\Psi$ is the nonclassical angular flux, $c$ is the scattering ratio, and $Q$ is an isotropic source.
Here, $P(\uom'\cdot\uom)d\Omega$ represents the probability that when a particle with direction of flight $\uom'$ scatters, its outgoing direction of flight will lie in $d\Omega$ about $\uom$.
This equation is subject to the incident boundary angular flux \cite{larfra_17}
\bal
\Psi(\ux,\uom,s) = \Psi^{b}(\ux,\uom)\delta(s),\quad \ux\in\partial V, \; \un\cdot\uom < 0,\; 0<s.
\nal
\nsub
The angular-dependent nonclassical total cross section $\Sigma_t(\uom,s)$ in \cref{1a} satisfies
\bal\label{2}
p(\uom,s) = \Sigma_t(\uom,s)e^{-\int_0^s \Sigma_t(\uom,s')ds'},
\nal
where $p(\uom,s)$ is the free-path distribution function in the direction $\uom$.

If classical transport takes place, $\Sigma_t$ is independent of both $\uom$ and $s$.
In this case, the free-path distribution reduces to the exponential $p(s) = \Sigma_te^{-\Sigma_ts}$, and \cref{1} reduce to the classical linear Boltzmann equation
\bsub
\bal\label[pluraleq]{3}
\uom\cdot\unab\Psi_c(\ux,\uom) + \Sigma_t\Psi_c (\ux,\uom) &= c\int_{4\pi}P(\uom'\cdot\uom)\Sigma_t\Psi_c(\ux,\uom')d\Omega' + \f{Q(\ux)}{4\pi},\\
& \hspace{200pt} \ux\in V,\; \uom\in 4\pi,\nonumber\\
\Psi_c(\ux,\uom) &= \Psi^{b}(\ux,\uom),\quad \ux\in\partial V, \;\un\cdot\uom < 0,
\nal
for the classical angular flux
\bal\label{3c}
\Psi_c(\ux,\uom) = \int_0^\infty\Psi(\ux,\uom,s)ds.
\nal
\nsub

Numerical results for the nonclassical theory have been provided for diffusion-based approximations and for moment models in the diffusive regime \cite{vaslar_09, larvas_11, vas_13, kryber_13, vaslar_14b, vassla_17a}.
To our knowledge, numerical results for the nonclassical transport equation given by \cref{1} are only available for problems in rod geometry \cite{vaskry_15, vassla_16, vaskry_17}.
This is in part due to the difficult task of estimating the nonclassical free-path distribution.
Another reason is that, given the $s$-dependence of $\Sigma_t$ and the improper integral on the right-hand side of \cref{1}, a direct deterministic approach that involves discretizing the variable $s$ is inefficient.

The goal of this paper is to introduce an approach to numerically solve \cref{1} in a deterministic fashion, using available methods.
We combine the multiple collision formalism \cite{papzec_72} and a spectral approach to obtain a set of coupled differential equations that can be solved recursively.
These equations have the form of a purely absorbing \textit{classical} transport equation with a fixed (known) source, and can be solved by any traditional method. 
Here, we present numerical results to the one-dimensional (1-D) nonclassical transport equation in slab geometry under both classical and nonclassical assumptions.
We use the LTS$_N$ method \cite{segvil_99} to solve the set of classical equations.
These results show \#\#\#\#\# THAT EVERYTHING WORKS \#\#\#\#\#

The remainder of this paper is organized as follows.
\#\#\#\#\#\#\#\#\#\#

\section{The Proposed Method\\ (looking for better section title)}\label{sec2}
\setcounter{section}{2}
%\setcounter{equation}{0} 

We consider \cref{1a} in an equivalent ``initial value'' form:
\bsub\label[pluraleq]{4}
\bal
&\f{\partial }{\partial s}\Psi(\ux,\uom,s) + \uom\cdot\unab\Psi(\ux,\uom,s) + \Sigma_t(\uom,s)\Psi (\ux,\uom,s) = 0,\\
%\,\, \ux\in V,\,\, \uom\in 4\pi,\,\, 0<s,\\
& \Psi(\ux,\uom,0)=c\int_{4\pi}\int_0^{\infty}P(\uom'\cdot\uom)\Sigma_t(\uom',s')\Psi(\ux,\uom',s')d\Omega' ds' + \f{Q(\ux)}{4\pi},
%\,\, \ux\in V,\,\, \uom\in 4\pi,
\nal
\nsub
and define $\psi$ such that
\bal\label{5}
\Psi(\ux,\uom,s)\equiv \psi(\ux,\uom,s)e^{-\int_0^s \Sigma_t(\uom,s')ds'}.
\nal
We can now rewrite the nonclassical problem as
\bsub\label[pluraleq]{6}
\bal
&\f{\partial }{\partial s}\psi(\ux,\uom,s) + \uom\cdot\unab\psi(\ux,\uom,s) = 0,\label{6a}\\
%\,\, \ux\in V,\,\, \uom\in 4\pi,\,\, 0<s,\label{6a}\\
& \psi(\ux,\uom,0)= S(\ux,\uom)+\f{Q(\ux)}{4\pi},\label{6b}\\
%\,\, \ux\in V,\,\, \uom\in 4\pi,\label{6b}\\
& \psi(\ux,\uom,s) =  \Psi^{b}(\ux,\uom)\delta(s)e^{\int_0^s\Sigma_t(\uom,s')ds'},\quad \ux\in\partial V,\; \un \cdot \uom <0,
\nal
where
\bal
S(\ux,\uom) = c\int_{4\pi}\int_0^{\infty} P(\uom'\cdot\uom)p(\uom',s')\psi(\ux,\uom',s')d\Omega' ds'.
\nal
\nsub

Using the theory of multiple collisions \cite{papzec_72}, we define
\bal\label{7}
\psi(\ux,\uom,s) = \sum_{k=0}^\infty \psi^{(k)}(\ux,\uom,s),
\nal
where $\psi^{(k)}$ represents the component of the angular flux consisting of particles that have undergone \textit{exactly} $k$ collisions.
It is easy to see that $\psi^{(k)}$ satisfies
\bsub\label[pluraleq]{8}
\bal
&\f{\partial }{\partial s}\psi^{(k)}(\ux,\uom,s) + \uom\cdot\unab\psi^{(k)}(\ux,\uom,s) = 0,\quad k=0,1,2,...\,,\label{8a}\\
& \psi^{(0)}(\ux,\uom,0)= \f{Q(\ux)}{4\pi},\label{8b}\\
&\psi^{(0)}(\ux,\uom,s) =  \Psi^{b}(\ux,\uom)\delta(s)e^{\int_0^s\Sigma_t(\uom,s')ds'},\quad \ux\in\partial V,\; \un \cdot \uom <0,\label{8c}\\
&\psi^{(k)}(\ux,\uom,0)= S^{(k-1)}(\ux,\uom), \quad k=1,2,...\,,\label{8d}\\
%& \psi^{(k)}(\ux,\uom,0)= F^{(k)}(\ux,\uom) = \left\{\begin{array}{cc}
%\f{Q(\ux)}{4\pi}, & k=0,\\
%S^{(k-1)}(\ux,\uom), & k > 0,
%\end{array}
%\right. \label{8b}\\
&\psi^{(k)}(\ux,\uom,s) =  0,\quad \ux\in\partial V,\; \un \cdot \uom <0,\; k=1,2,...\,,\label{8e}
\nal
\nsub
where $S^{(k-1)}(\ux,\uom) = c\int_{4\pi}\int_0^{\infty} P(\uom'\cdot\uom)p(\uom',s')\psi^{(k-1)}(\ux,\uom',s')d\Omega' ds'$.

To apply the spectral method, we approximate $\psi^{(k)}$ by a truncated series of Laguerre polynomials \cite{hoc_72} in $s$:
\bal\label{9}
\psi^{(k)}(\ux,\uom,s) = \sum_{m=0}^{M} \psi^{(k)}_m(\ux,\uom)L_m(s),\quad k=0,1,2,...\,,
\nal
and replace this ansatz into \cref{8}.
The Laguerre polynomials $\{ L_0(s), L_1(s), ..., L_M(s)\}$ are orthogonal with respect to the weight function $e^{-s}$, and satisfy $\f{d}{ds}L_m(s) = \left(\f{d}{ds}-1\right)L_{m-1}(s)$ for $m>0$. 
Therefore, multiplying \cref{8a,8c,8e} by $e^{-s}L_m(s)$ and operating on them by $\int_0^\infty (\cdot)ds$, we obtain
\bsub\label{10}
\bal
&\uom\cdot\unab\psi_m^{(k)}(\ux,\uom) = \sum_{j=m+1}^M \psi_j^{(k)}(\ux,\uom), \quad m=0,1,...,M,\; k=0,1,2,...\,,\\
&\psi_m^{(0)}(\ux,\uom) =  \Psi^{b}(\ux,\uom),\quad \ux\in\partial V,\; \un \cdot \uom <0,\; m=0,1,...,M,\\
&\psi_m^{(k)}(\ux,\uom) =  0,\quad \ux\in\partial V,\; \un \cdot \uom <0,\; m=0,1,...,M, \; k=1,2,...\,.
\nal
\nsub
Moreover, \cref{8b,8d} respectively yield
\bsub\label{11}
\bal
\sum_{j=m+1}^M\psi^{(0)}_j(\ux,\uom) &= \f{Q(\ux)}{4\pi}  - \sum_{j=0}^m\psi_j^{(0)}(\ux,\uom),\\
\sum_{j=m+1}^M\psi^{(k)}_j(\ux,\uom) &= S^{(k-1)}(\ux,\uom)  - \sum_{j=0}^m\psi_j^{(k)}(\ux,\uom), \quad k=1,2,...\, .
\nal
\nsub

Next, we define $U_m^{(k)}(\ux,\uom)$ as
\bsub\label[pluraleq]{12}
\bal
U^{(0)}_0(\ux,\uom) &= \f{Q(\ux)}{4\pi}, \\
U^{(k)}_0(\ux,\uom) &= S^{(k-1)}(\ux,\uom),\quad k=1,2,...\,,\\
U^{(k)}_m(\ux,\uom) &= U^{(k)}_{m-1}(\ux,\uom) - \psi^{(k)}_{m-1}(\ux,\uom), \quad m=1,...,M,\; k=0,1,2,...\,.
\nal
\nsub
Finally, using \cref{10,11,12}, we can rewrite the nonclassical problem as a set of coupled differential equations:
\bsub\label[pluraleq]{13}
\bal
&\uom\cdot\unab\psi_m^{(k)}(\ux,\uom) + \psi_m^{(k)}(\ux,\uom)  = U_m^{(k)}(\ux,\uom),\quad m=0,1,...,M,\; k=0,1,2,...\,,\label{13a}\\
&\psi_m^{(0)}(\ux,\uom) =  \Psi^{b}(\ux,\uom),\quad \ux\in\partial V,\; \un \cdot \uom <0,\quad m=0,1,...,M,\label{13b}\\
&\psi_m^{(k)}(\ux,\uom) =  0,\quad \ux\in\partial V,\; \un \cdot \uom <0,\; m=0,1,...,M, \; k=1,2,...\,\label{13c}.
\nal
\nsub

\Cref{13} can be solved recursively using any homogeneous solver.
Starting at $k=0$, each $\psi^{(k)}$ is attained as follows:  
\begin{enumerate}
\item $m=0$;
\item While $m < M$
\begin{itemize}
\item[2.1.] Solve \cref{13} for $\psi_m^{(k)}$, using the fact that $U_m^{(k)}$ is a known function given by \cref{12};
\item[2.2.] $m=m+1$;
\end{itemize}
\item Use \cref{9} to obtain $\psi^{(k)}$;
\item Repeat for $k=k+1$.
\end{enumerate}
Using a stopping criterion for the $k$ iterations, the nonclassical angular flux $\Psi$ is recovered from \cref{5,7}.
Finally, the angular flux $\Psi_c(\ux,\uom)$ is obtained using \cref{3c}.   

\#\#\#\#\#\#\#\#\#\#

\#\#\#\#\#\#\#\#\#\#

DISCUSSION: APPROXIMATIONS, SOLVERS, COMMENTS ON CONVERGENCE, ETC.

\#\#\#\#\#\#\#\#\#\#

\#\#\#\#\#\#\#\#\#\# 
\\

\#\#\#\#\#\#\#\#\#\#

\#\#\#\#\#\#\#\#\#\# 

NUMERICAL APPROACH WE WILL PRESENT; 1-D SLAB 

\#\#\#\#\#\#\#\#\#\#

\#\#\#\#\#\#\#\#\#\# 

\section{Validation with Classical Transport}\label{sec4}
\setcounter{section}{4}
%\setcounter{equation}{0} 

WE SHOW RESULTS FOR 1-D SLAB (???AND 3D???) TEST PROBLEMS WITH CLASSICAL TRANSPORT\\

\#\#\#\#\#\#\#\#\#\#

\#\#\#\#\#\#\#\#\#\#


\section{Nonclassical Results}\label{sec5}
\setcounter{section}{5}
%\setcounter{equation}{0} 

WE SHOW RESULTS FOR 1-D SLAB TEST PROBLEMS WITH NONCLASSICAL TRANSPORT -- E.G., RANDOM PERIODIC\\

\#\#\#\#\#\#\#\#\#\#

\#\#\#\#\#\#\#\#\#\#

\section{Conclusion}\label{sec6}
\setcounter{section}{6}
%\setcounter{equation}{0} 

\#\#\#\#\#\#\#\#\#\#

\#\#\#\#\#\#\#\#\#\#

\section*{Comments (for our discussion)}
\setcounter{section}{4}
%\setcounter{equation}{0} 

\begin{itemize}
\item While we do not explicitly discretize $s$, the following takes place:
\begin{itemize}
\item $\psi(x,\mu,s)$ is approximated by a truncated series of Laguerre polynomials in $s$
\item The integral in $s$ described in \cref{6} will probably need to be performed numerically
\item At the end of the algorithm, we need to calculate
$$\hat\Psi(\ux,\uom) = \int_0^\infty \Psi(\ux,\uom,s)ds =  \int_0^{\infty} \left(e^{-\int_0^s\Sigma_t(\uom,s')ds'}\sum_{k=0}^{K}\sum_{m=0}^M\psi^{(k)}_m(\ux,\uom)L_m(s)\right)ds,$$
which will also need to be performed numerically 
\item The LTS$_N$ matrix should be simple and easy; since it's purely absorbing, $A$ is a diagonal matrix $1/\mu_n$
\end{itemize}
\item The convolution integrals in LTS$_N$ will \textit{probably} need to be solved numerically due to the recursiveness of the problem arising from the source term. \textbf{Is there a way to do that analitically?}
\item Due to the source iteration approach, convergence will be slow as problems become more diffusive.
\item It is not clear to me what will be the more time-consuming step. My guess is that the time to converge the source iteration will dominate in diffusive problems; for absorbing problems, I do not know.   
\item Nonclassical boundary conditions are tricky because of the $\delta(s)$. I \textit{expect} the method here to work, but there may be convergence problems due to the Laguerre approximation/truncation. We'll need to test it for a few problems and see what we get; I'll work on figuring out the analytical convergence.
\textit Still regarding the boundary conditions, it is possible that the best solution will be using the \textit{forward} nonclassical equation. That, however, is beyond the current scope.
\item To validate the method, we will apply the algorithm to solve classical problems. In that case, $p(\mu,s) = \Sigma_te^{-\Sigma_t s}$. This will also allow us to see how efficiently the method is.
\item After it is validated, we can apply the algorithm to the random periodic case we have been working on; in the future, we can go for general stochastic mixtures.
\end{itemize}

\section*{References}

\bibliography{ANE-16}












\end{document}
